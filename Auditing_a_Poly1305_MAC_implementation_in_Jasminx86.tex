\documentclass[twocolumn]{article}
\usepackage[utf8]{inputenc}
\usepackage[a4paper, total={7in, 9in}]{geometry}
%\usepackage{graphicx}
%\usepackage{float}
\usepackage{amsmath}
\usepackage[
backend=biber,
style=numeric
]{biblatex}

\addbibresource{cryptography.bib}

\begin{document}
\twocolumn[{    
    \center
    \textsc{\Large Universidade do Minho} \\ [0.5cm]
    \textsc{\Large Mestrado em Engenharia Informática} \\ [0.5cm]
    \textsc{\large Tecnologia Criptográfica} \\ [0.5cm]
    
    {\LARGE \bfseries Auditing a Poly1305 MAC implementation in Jasmin for x86} \\ [0.5cm]

    \begin{tabular}{c}
        Miguel Miranda Quaresma \\
        A77049 \\
    \end{tabular} \\ [0.5cm]

    \today \\ [1cm]
}]

\begin{abstract}
    Poly1305 is a one time authenticator developed with performance in mind, that generates a message authentication code for a given input and secret key. Jasmin is 
    framework for develpoing high performance and high assurance cryptographic software. The current works aims to examine an implementation of the Poly1305 MAC using
    the Jasmin framework. The Poly1305 MAC is described in at a high abstraction level, followed by an in-depth analysis of the Jasmin implementation of the algorithm.
    The work concludes with an auditing/verifcation of the premisses/assumptions that were made for the implementation.
    \end{abstract}

\section{Introduction}
Message authentication codes play a major role in guaranteeing the authenticity and integrity of data being sent across an untrusted channel. There are many 
cryptoprimitives that work as MAC's and, for a long time, HMAC(Hash-MACs) were preferred over other MACs due to their performance, with other primitives such 
as the ones based on universal hashing being discarded. This was further reinforced by the introduction of assembly instructions that performed hash functions 
directly in hardware, such as Intel's instruction for SHA-256: \textbf{INSERT x86 INSTRUCTION HERE}. The development of cryptoprimitives such as Poly1305 MAC 
using Jasmin, a framework for developing high performance and high assurance cryptographic software \cite{jasmin_paper}, allowed this tendency to be reversed 
by obtaining highly performant implementations with relative ease.

\section{Poly1305 explained}
Poly1305 is a message authentication code(MAC) that guarantees integrity and authenticity of messages. Poly1305 works similarly to a 
universal hash function, evaluating a given message over a polynomial and using the result as a MAC. Additionaly, Poly1305 evaluates this
polynomial over a prime field, from 0 to $2^{130}-5$, where the name (Poly\textbf{1305}) stems from. It uses a 256 bit secret, derived via a Password-Based Key Derivation
Function(PBDKF), using the first 128 bits for the key, k and r for calculating the polynomial.

Poly1305 works by breaking the input in 16 byte blocks, appending each one with a 1 byte(00000001) to prevent forgery, via padding attacks. It then proceeds to apply the 
following algorithm/formula:

\begin{verbatim}
h = 0
for block in blocks:
    h += block
    h *= r
h+=k mod 2^130-5
\end{verbatim}

where:
\begin{itemize}
    \item \texttt{block} is a 17 byte block from the message 
    \item \texttt{r} and \texttt{k} are 128 bit values derived from the 256 bit key used by Poly1305
\end{itemize}


\section{Conclusion}

\printbibliography

\end{document}
