\documentclass[twocolumn]{article}
\usepackage[utf8]{inputenc}
\usepackage[a4paper, total={7in, 9in}]{geometry}
%\usepackage{graphicx}
%\usepackage{float}
\usepackage{amsmath}
\usepackage[
backend=biber,
style=numeric
]{biblatex}

\addbibresource{cryptography.bib}

\begin{document}
\twocolumn[{    
    \center
    \textsc{\Large Universidade do Minho} \\ [0.5cm]
    \textsc{\Large Mestrado em Engenharia Informática} \\ [0.5cm]
    \textsc{\large Tecnologia Criptográfica} \\ [0.5cm]
    
    {\LARGE \bfseries Auditing a Poly1305 MAC implementation in Jasmin for x86} \\ [0.5cm]

    \begin{tabular}{c}
        Miguel Miranda Quaresma \\
        A77049 \\
    \end{tabular} \\ [0.5cm]

    \today \\ [1cm]
}]

\begin{abstract}
    Current cryptograghic systems all share one common pitfall: lack of focus in performance. This is motivated by the fact that speed isn't always seen as a primary concern
    for highly critical systems where the security requirements are extremely high. However, the rise in the use of low powered devices such as IoT, smartphones, smartcards,
    etc has incentivized the investment in highly performant cryptographic systems that can be used in such devices. The present work aims to audit a component of such 
    a system, namely an implementation of the Poly1305 MAC using Jasmin, a framework for developing high performance and high assurance cryptographic 
    software(\cite{jasmin_paper}).
\end{abstract}

\section{Introduction}
Poly1305 is one time authenticator developed with performance in mind(\cite{poly1305_aes_bernstein}) that generates a message authentication code for a given input. 
When combined with a framework for developing high performance and high assurance cryptographic software such as Jasmin, the result is a high performance tool that 
can be used in devices with low computational power such as IoT, Smartphones, Smart-cards, etc.

\section{Poly1305 explained}
Poly1305 is a message authentication code(MAC) that guarantees integrity and authenticity of messages. Poly1305 works similarly to a 
universal hash function, evaluating a given message over a polynomial and using the result as a MAC. Additionaly, Poly1305 evaluates this
polynomial over a prime field, from 0 to $2^{130}-5$, where the name (Poly\textbf{1305}) stems from. It uses a 256 bit secret, derived via a Password-Based Key Derivation
Function(PBDKF), using the first 128 bits for the key, k and r for calculating the polynomial.

Poly1305 works by breaking the input in 16 bytes blocks, appending each one with a 1 byte(00000001) to prevent forgery. It then proceeds to apply the following 
algorithm/formula:

\begin{verbatim}
h = 0
for block in blocks:
    h += block
    h *= r
h+=k mod 2^130-5
\end{verbatim}

where:
\begin{itemize}
    \item \texttt{block} is a 17 byte block from the message 
    \item \texttt{r} and \texttt{k} are 128 bit values derived from the 256 bit key used by Poly1305
\end{itemize}


\section{Conclusion}

\printbibliography

\end{document}
